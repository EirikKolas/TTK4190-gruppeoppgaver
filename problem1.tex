\section*{Problem 1 - Attitude Control of Satellite}



\subsection*{Problem 1.1} 

\begin{equation}
\label{eq:dynamics}		% The label is used when referring to this equation. 
	\begin{aligned}
		\dot{\mathbf{q}} = \mathbf{T}_q (\mathbf{q} ) \boldsymbol{\omega} \\
		\mathbf{I}_{CG} \dot{\boldsymbol{\omega}} - \mathbf{S} (\mathbf{I}_{CG} \boldsymbol{\omega} ) \boldsymbol{\omega} & =  \boldsymbol{\tau}
	\end{aligned}	
\end{equation}

Rewriting equation \ref{eq:dynamics}, we can find the equilibrium points for the Sattelite system by solving equation \ref{eq:dynamics2} given the  initial values. 

\begin{equation}
\label{eq:dynamics2}		% The label is used when referring to this equation. 
	\begin{aligned}
		\dot{\mathbf{q}} = \mathbf{T}_q (\mathbf{q} ) \boldsymbol{\omega} \\
		\dot{\boldsymbol{\omega}} =  \frac{1}{\mathbf{I}_{CG}} (\boldsymbol{\tau} + \mathbf{S} (\mathbf{I}_{CG} \boldsymbol{\omega} ) \boldsymbol{\omega} &)
	\end{aligned}	
\end{equation}

\begin{equation}
    \begin{bmatrix}
        \dot{\epsilon_1} \\ \dot{\epsilon_2} \\ \dot{\epsilon_3}
    \end{bmatrix}
    = \frac{1}{2}
      \begin{bmatrix}
        \eta & -\epsilon_3 & \epsilon_2 \\ \epsilon_3 & \eta & - \epsilon_1 \\ -\epsilon_2 & \epsilon_1 & \eta
    \end{bmatrix} 
    \omega
\end{equation}

\begin{equation}
    \begin{bmatrix}
    \dot{p} \\ \dot{q} \\ \dot{r}
    \end{bmatrix}
    =
    \begin{bmatrix}
    \frac{\tau_1}{mR^2} \\ \frac{\tau_2}{mR^2} \\ \frac{\tau_3}{mR^2} 
    \end{bmatrix}
\end{equation}




Inserting the initial values into equations 3 and 4, the calculated equilibrium points become the following: 
\begin{align*}
    \mathbf{x}_0 =
    \begin{bmatrix}
        \epsilon_1 \\ \epsilon_2 \\ \epsilon_3 \\p \\ q \\ r 
    \end{bmatrix}
    = 
    \begin{bmatrix}
        0 \\ 0 \\ 0 \\ 0 \\ 0 \\ 0
    \end{bmatrix}
\end{align*}

Linearizing the spacecraft model around $x_0$: 

\begin{equation}
    \dot{\mathbf{x}} = \mathbf{f(x)}
    \begin{bmatrix}
        \frac{1}{2}(\eta \mathbf{p} -\epsilon_3 \mathbf{q} + \epsilon_2 \mathbf{r}) \\
        \frac{1}{2}(\epsilon_3 \mathbf{p} + \eta \mathbf{q}  - \epsilon_1 \mathbf{r})\\
        \frac{1}{2}(-\epsilon_2 \mathbf{p} +\epsilon_1 \mathbf{q} + \eta \mathbf{r}) \\
        \frac{\tau_1}{mR^2}\\
        \frac{\tau_2}{mR^2}\\
        \frac{\tau_3}{mR^2}\\
    \end{bmatrix} 
\end{equation}


\begin{equation}
    \dot{x} = \frac{df}{dx_{|x_0}}(x-x_0) + \frac{df}{du_{|u_0}}(u-u_0)
\end{equation}

Calculating the jacobians with the initial values results in the linearized model:

\begin{equation}
    \dot{x} = 
    \begin{bmatrix}
        0 & 0 & 0 & \frac{1}{2}  & 0 & 0 \\
        0 & 0 & 0 & 0 & \frac{1}{2}  & 0 \\
        0 & 0 & 0 & 0 & 0 & \frac{1}{2}  \\
        0 & 0 & 0 & 0 & 0 & 0 \\
        0 & 0 & 0 & 0 & 0 & 0 \\
        0 & 0 & 0 & 0 & 0 & 0 \\
    \end{bmatrix}
    x + 
    \begin{bmatrix}
        0 & 0 & 0 \\
        0 & 0 & 0 \\
        0 & 0 & 0 \\
        \frac{1}{720} & 0 & 0 \\
        0 & \frac{1}{720}  & 0 \\
        0 & 0 & \frac{1}{720} \\
    \end{bmatrix}
    u
\end{equation}








\subsection*{Problem 1.2}

The system should have critical damping avoiding oscillations of the sattelite



\subsubsection*{Inner Section 1}
\emph{text..}

\subsubsection*{Inner Section 2}
...

\subsection*{Problem 1.3}
Answer Problem 1.3 here. Equation (2) from the assignment can be written as: 
\begin{equation}
  \label{eq:tau}
  \mathbf{\tau} = -\mathbf{K}_d \boldsymbol{\omega} - k_p \boldsymbol{\epsilon}
\end{equation}

\subsection*{Problem 1.4}
The quaternion error can be written as
 \begin{equation}
	 \tilde{\mathbf{q}} := \left[
	 \begin{array}{c}
		 \tilde{\eta} \\
		 \tilde{\epsilon}
	 \end{array}
	 \right] = \bar{\mathbf{q}}_d \otimes \mathbf{q} 
 \end{equation}

\subsection*{Problem 1.5}
In problems with simulations, you need to include figures in the report:
\begin{figure}[ht]
	\centering
	\includegraphics[width=0.7\textwidth]{fig1} % Filename is "fig1.png" and must be located in the same folder as this file. If you have a folder containing all the figures you can use "Figures/fig 1" as long as the "Figures" folder is placed in the same folder as this file.
	\caption{Figure of something useful.}
	\label{fig:fig1}
\end{figure}

You can now refer to this figure as \figref{fig:fig1}. You can also insert figures side-by-side as in Figure \ref{fig:2}. %Notice that \figref includes the word Figure before the reference. If you use "\ref", you need to write the word Figure yourself. 
\begin{figure}[ht]
	\centering
	\begin{subfigure}[b]{0.45\textwidth}
		\includegraphics[width=\textwidth]{fig1}
		\caption{caption..}
		\label{fig:2a}
	\end{subfigure}
	~ %add desired spacing between images, e. g. ~, \quad, \qquad, \hfill etc. 
	%(or a blank line to force the subfigure onto a new line)
	\begin{subfigure}[b]{0.45\textwidth}
		\includegraphics[width=\textwidth]{fig1}
		\caption{caption..}
		\label{fig:2b}
	\end{subfigure}
	\begin{subfigure}[b]{0.45\textwidth}
		\includegraphics[width=\textwidth]{fig1}
		\caption{caption..}
		\label{fig:2c}
	\end{subfigure}
	\begin{subfigure}[b]{0.45\textwidth}
		\includegraphics[width=\textwidth]{fig1}
		\caption{caption..}
		\label{fig:2d}
	\end{subfigure}		
	\caption{Caption for all figures}\label{fig:2}
\end{figure}


\subsection*{Problem 1.6}
The control law in this problem can be written as
\begin{equation}
	\boldsymbol{\tau} = -\mathbf{K}_d \tilde{\boldsymbol{\omega}} - k_p \tilde{\boldsymbol{\epsilon}}
\end{equation}
and the desired angular velocity as
\begin{equation}
	\boldsymbol{\omega}_d = \mathbf{T}^{-1}_{\Theta_d}(\Theta_d)\dot{\Theta}_d
\end{equation}

\subsection*{Problem 1.7}
The Lyapunov function can be written as 
 \begin{equation}
	 V = \frac{1}{2} \tilde{\boldsymbol{\omega}}^{\top} \mathbf{I}_{CG}\tilde{\boldsymbol{\omega}} + 2 k_p (1-\tilde{\eta})
 \end{equation}
and the derivative as 
\begin{equation}
	\dot{V} = -k_d \boldsymbol{\omega}^{\top} \boldsymbol{\omega}
\end{equation}

% Note that \mathbf can be used for bold letters in math mode (within equations and dollar signs). \boldsymbol can be used to get bold greek letters.  